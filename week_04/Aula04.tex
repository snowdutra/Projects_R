% Options for packages loaded elsewhere
\PassOptionsToPackage{unicode}{hyperref}
\PassOptionsToPackage{hyphens}{url}
%
\documentclass[
]{article}
\usepackage{amsmath,amssymb}
\usepackage{iftex}
\ifPDFTeX
  \usepackage[T1]{fontenc}
  \usepackage[utf8]{inputenc}
  \usepackage{textcomp} % provide euro and other symbols
\else % if luatex or xetex
  \usepackage{unicode-math} % this also loads fontspec
  \defaultfontfeatures{Scale=MatchLowercase}
  \defaultfontfeatures[\rmfamily]{Ligatures=TeX,Scale=1}
\fi
\usepackage{lmodern}
\ifPDFTeX\else
  % xetex/luatex font selection
\fi
% Use upquote if available, for straight quotes in verbatim environments
\IfFileExists{upquote.sty}{\usepackage{upquote}}{}
\IfFileExists{microtype.sty}{% use microtype if available
  \usepackage[]{microtype}
  \UseMicrotypeSet[protrusion]{basicmath} % disable protrusion for tt fonts
}{}
\makeatletter
\@ifundefined{KOMAClassName}{% if non-KOMA class
  \IfFileExists{parskip.sty}{%
    \usepackage{parskip}
  }{% else
    \setlength{\parindent}{0pt}
    \setlength{\parskip}{6pt plus 2pt minus 1pt}}
}{% if KOMA class
  \KOMAoptions{parskip=half}}
\makeatother
\usepackage{xcolor}
\usepackage[margin=1in]{geometry}
\usepackage{color}
\usepackage{fancyvrb}
\newcommand{\VerbBar}{|}
\newcommand{\VERB}{\Verb[commandchars=\\\{\}]}
\DefineVerbatimEnvironment{Highlighting}{Verbatim}{commandchars=\\\{\}}
% Add ',fontsize=\small' for more characters per line
\usepackage{framed}
\definecolor{shadecolor}{RGB}{248,248,248}
\newenvironment{Shaded}{\begin{snugshade}}{\end{snugshade}}
\newcommand{\AlertTok}[1]{\textcolor[rgb]{0.94,0.16,0.16}{#1}}
\newcommand{\AnnotationTok}[1]{\textcolor[rgb]{0.56,0.35,0.01}{\textbf{\textit{#1}}}}
\newcommand{\AttributeTok}[1]{\textcolor[rgb]{0.13,0.29,0.53}{#1}}
\newcommand{\BaseNTok}[1]{\textcolor[rgb]{0.00,0.00,0.81}{#1}}
\newcommand{\BuiltInTok}[1]{#1}
\newcommand{\CharTok}[1]{\textcolor[rgb]{0.31,0.60,0.02}{#1}}
\newcommand{\CommentTok}[1]{\textcolor[rgb]{0.56,0.35,0.01}{\textit{#1}}}
\newcommand{\CommentVarTok}[1]{\textcolor[rgb]{0.56,0.35,0.01}{\textbf{\textit{#1}}}}
\newcommand{\ConstantTok}[1]{\textcolor[rgb]{0.56,0.35,0.01}{#1}}
\newcommand{\ControlFlowTok}[1]{\textcolor[rgb]{0.13,0.29,0.53}{\textbf{#1}}}
\newcommand{\DataTypeTok}[1]{\textcolor[rgb]{0.13,0.29,0.53}{#1}}
\newcommand{\DecValTok}[1]{\textcolor[rgb]{0.00,0.00,0.81}{#1}}
\newcommand{\DocumentationTok}[1]{\textcolor[rgb]{0.56,0.35,0.01}{\textbf{\textit{#1}}}}
\newcommand{\ErrorTok}[1]{\textcolor[rgb]{0.64,0.00,0.00}{\textbf{#1}}}
\newcommand{\ExtensionTok}[1]{#1}
\newcommand{\FloatTok}[1]{\textcolor[rgb]{0.00,0.00,0.81}{#1}}
\newcommand{\FunctionTok}[1]{\textcolor[rgb]{0.13,0.29,0.53}{\textbf{#1}}}
\newcommand{\ImportTok}[1]{#1}
\newcommand{\InformationTok}[1]{\textcolor[rgb]{0.56,0.35,0.01}{\textbf{\textit{#1}}}}
\newcommand{\KeywordTok}[1]{\textcolor[rgb]{0.13,0.29,0.53}{\textbf{#1}}}
\newcommand{\NormalTok}[1]{#1}
\newcommand{\OperatorTok}[1]{\textcolor[rgb]{0.81,0.36,0.00}{\textbf{#1}}}
\newcommand{\OtherTok}[1]{\textcolor[rgb]{0.56,0.35,0.01}{#1}}
\newcommand{\PreprocessorTok}[1]{\textcolor[rgb]{0.56,0.35,0.01}{\textit{#1}}}
\newcommand{\RegionMarkerTok}[1]{#1}
\newcommand{\SpecialCharTok}[1]{\textcolor[rgb]{0.81,0.36,0.00}{\textbf{#1}}}
\newcommand{\SpecialStringTok}[1]{\textcolor[rgb]{0.31,0.60,0.02}{#1}}
\newcommand{\StringTok}[1]{\textcolor[rgb]{0.31,0.60,0.02}{#1}}
\newcommand{\VariableTok}[1]{\textcolor[rgb]{0.00,0.00,0.00}{#1}}
\newcommand{\VerbatimStringTok}[1]{\textcolor[rgb]{0.31,0.60,0.02}{#1}}
\newcommand{\WarningTok}[1]{\textcolor[rgb]{0.56,0.35,0.01}{\textbf{\textit{#1}}}}
\usepackage{longtable,booktabs,array}
\usepackage{calc} % for calculating minipage widths
% Correct order of tables after \paragraph or \subparagraph
\usepackage{etoolbox}
\makeatletter
\patchcmd\longtable{\par}{\if@noskipsec\mbox{}\fi\par}{}{}
\makeatother
% Allow footnotes in longtable head/foot
\IfFileExists{footnotehyper.sty}{\usepackage{footnotehyper}}{\usepackage{footnote}}
\makesavenoteenv{longtable}
\usepackage{graphicx}
\makeatletter
\def\maxwidth{\ifdim\Gin@nat@width>\linewidth\linewidth\else\Gin@nat@width\fi}
\def\maxheight{\ifdim\Gin@nat@height>\textheight\textheight\else\Gin@nat@height\fi}
\makeatother
% Scale images if necessary, so that they will not overflow the page
% margins by default, and it is still possible to overwrite the defaults
% using explicit options in \includegraphics[width, height, ...]{}
\setkeys{Gin}{width=\maxwidth,height=\maxheight,keepaspectratio}
% Set default figure placement to htbp
\makeatletter
\def\fps@figure{htbp}
\makeatother
\setlength{\emergencystretch}{3em} % prevent overfull lines
\providecommand{\tightlist}{%
  \setlength{\itemsep}{0pt}\setlength{\parskip}{0pt}}
\setcounter{secnumdepth}{-\maxdimen} % remove section numbering
\ifLuaTeX
  \usepackage{selnolig}  % disable illegal ligatures
\fi
\usepackage{bookmark}
\IfFileExists{xurl.sty}{\usepackage{xurl}}{} % add URL line breaks if available
\urlstyle{same}
\hypersetup{
  pdftitle={Aula04},
  hidelinks,
  pdfcreator={LaTeX via pandoc}}

\title{Aula04}
\author{}
\date{\vspace{-2.5em}}

\begin{document}
\maketitle

\{r setup, include=FALSE\} knitr::opts\_chunk\$set(echo = TRUE)

\subsection{Exercícios}\label{exercicios}

\begin{enumerate}
\def\labelenumi{\arabic{enumi})}
\tightlist
\item
  Use o comando select() para criar um subconjunto dos dados que inclua
  somente as colunas: escolaridade (educational level), se o respondente
  tem algum financiamento educacional (educational loan), situação de
  trabalho (employment status), e aprovação ao Governo (Trump approval).
  Apresente o objeto. Dica: consulte o dicionário de variáveis para
  identificar as variáveis corretas.
\end{enumerate}

\begin{Shaded}
\begin{Highlighting}[]
\FunctionTok{library}\NormalTok{(tidyverse)}
\FunctionTok{library}\NormalTok{(knitr)}


\CommentTok{\# Carregando e ajustando os dados}
\NormalTok{cces }\OtherTok{=} \FunctionTok{read\_csv}\NormalTok{(}\StringTok{"cces\_sample.csv"}\NormalTok{)}
\NormalTok{cces}\SpecialCharTok{$}\NormalTok{edloan[}\FunctionTok{is.na}\NormalTok{(cces}\SpecialCharTok{$}\NormalTok{edloan)] }\OtherTok{=} \DecValTok{2}
\NormalTok{cces1 }\OtherTok{=} \FunctionTok{drop\_na}\NormalTok{(cces)}

\CommentTok{\# Criando o subconjunto com as colunas solicitadas}
\NormalTok{cces\_subset }\OtherTok{=}\NormalTok{ cces1 }\SpecialCharTok{\%\textgreater{}\%} 
  \FunctionTok{select}\NormalTok{(educ, edloan, employ, CC18\_308a)}

\CommentTok{\# Apresentando o objeto}
\NormalTok{cces\_subset}
\end{Highlighting}
\end{Shaded}

\begin{verbatim}
## # A tibble: 969 x 4
##     educ edloan employ CC18_308a
##    <dbl>  <dbl>  <dbl>     <dbl>
##  1     2      2      5         2
##  2     6      2      1         4
##  3     3      2      1         4
##  4     5      2      5         4
##  5     2      2      8         4
##  6     3      2      1         4
##  7     2      2      5         1
##  8     3      2      7         1
##  9     5      1      1         4
## 10     5      2      1         4
## # i 959 more rows
\end{verbatim}

\begin{enumerate}
\def\labelenumi{\arabic{enumi})}
\setcounter{enumi}{1}
\tightlist
\item
  Use o comando recode() para criar uma nova coluna no objeto anterior
  chamada ``trump\_approve\_disapprove'' que recodifica a variável
  ``President Trump's job approval''. Um valor igual a ``1'' deve
  significar que o respondente ou ``strongly'' ou ``somewhat'' aprova o
  Governo, e o valor ``0'' deve significar que o respondente ou
  ``strongly'' ou ``somewhat'' desaprova o Governo Trump. Apresente os
  resultados do objeto no console.
\end{enumerate}

\begin{Shaded}
\begin{Highlighting}[]
\FunctionTok{library}\NormalTok{(tidyverse)}

\CommentTok{\# Carregando e ajustando os dados}
\NormalTok{cces }\OtherTok{=} \FunctionTok{read\_csv}\NormalTok{(}\StringTok{"cces\_sample.csv"}\NormalTok{)}
\end{Highlighting}
\end{Shaded}

\begin{verbatim}
## Rows: 1000 Columns: 25
## -- Column specification --------------------------------------------------------
## Delimiter: ","
## dbl (25): caseid, region, gender, educ, edloan, race, hispanic, employ, mars...
## 
## i Use `spec()` to retrieve the full column specification for this data.
## i Specify the column types or set `show_col_types = FALSE` to quiet this message.
\end{verbatim}

\begin{Shaded}
\begin{Highlighting}[]
\NormalTok{cces}\SpecialCharTok{$}\NormalTok{edloan[}\FunctionTok{is.na}\NormalTok{(cces}\SpecialCharTok{$}\NormalTok{edloan)] }\OtherTok{=} \DecValTok{2}
\NormalTok{cces1 }\OtherTok{=} \FunctionTok{drop\_na}\NormalTok{(cces)}

\CommentTok{\# Criando o subconjunto com as colunas solicitadas}
\NormalTok{cces\_subset }\OtherTok{=}\NormalTok{ cces1 }\SpecialCharTok{\%\textgreater{}\%} 
  \FunctionTok{select}\NormalTok{(educ, edloan, employ, CC18\_308a)}

\CommentTok{\# Criando a nova coluna trump\_approve\_disapprove com recode()}
\NormalTok{cces\_subset }\OtherTok{=}\NormalTok{ cces\_subset }\SpecialCharTok{\%\textgreater{}\%}
  \FunctionTok{mutate}\NormalTok{(}\AttributeTok{trump\_approve\_disapprove =} \FunctionTok{recode}\NormalTok{(CC18\_308a,}
                                           \StringTok{\textasciigrave{}}\AttributeTok{1}\StringTok{\textasciigrave{}} \OtherTok{=} \DecValTok{1}\NormalTok{,  }\CommentTok{\# Strongly approve {-}\textgreater{} 1}
                                           \StringTok{\textasciigrave{}}\AttributeTok{2}\StringTok{\textasciigrave{}} \OtherTok{=} \DecValTok{1}\NormalTok{,  }\CommentTok{\# Somewhat approve {-}\textgreater{} 1}
                                           \StringTok{\textasciigrave{}}\AttributeTok{3}\StringTok{\textasciigrave{}} \OtherTok{=} \DecValTok{0}\NormalTok{,  }\CommentTok{\# Somewhat disapprove {-}\textgreater{} 0}
                                           \StringTok{\textasciigrave{}}\AttributeTok{4}\StringTok{\textasciigrave{}} \OtherTok{=} \DecValTok{0}\NormalTok{)) }\CommentTok{\# Strongly disapprove {-}\textgreater{} 0}

\CommentTok{\# Apresentando o resultado no console}
\NormalTok{cces\_subset}
\end{Highlighting}
\end{Shaded}

\begin{verbatim}
## # A tibble: 969 x 5
##     educ edloan employ CC18_308a trump_approve_disapprove
##    <dbl>  <dbl>  <dbl>     <dbl>                    <dbl>
##  1     2      2      5         2                        1
##  2     6      2      1         4                        0
##  3     3      2      1         4                        0
##  4     5      2      5         4                        0
##  5     2      2      8         4                        0
##  6     3      2      1         4                        0
##  7     2      2      5         1                        1
##  8     3      2      7         1                        1
##  9     5      1      1         4                        0
## 10     5      2      1         4                        0
## # i 959 more rows
\end{verbatim}

\begin{Shaded}
\begin{Highlighting}[]
\FunctionTok{kable}\NormalTok{(}\FunctionTok{head}\NormalTok{(cces\_subset))}
\end{Highlighting}
\end{Shaded}

\begin{longtable}[]{@{}rrrrr@{}}
\toprule\noalign{}
educ & edloan & employ & CC18\_308a & trump\_approve\_disapprove \\
\midrule\noalign{}
\endhead
\bottomrule\noalign{}
\endlastfoot
2 & 2 & 5 & 2 & 1 \\
6 & 2 & 1 & 4 & 0 \\
3 & 2 & 1 & 4 & 0 \\
5 & 2 & 5 & 4 & 0 \\
2 & 2 & 8 & 4 & 0 \\
3 & 2 & 1 & 4 & 0 \\
\end{longtable}

\begin{enumerate}
\def\labelenumi{\arabic{enumi})}
\setcounter{enumi}{2}
\tightlist
\item
  Use summarise() para criar um sumário dos respondentes que estão
  empregados em tempo integral e são casados. A tabela deve apresentar a
  média e a mediana da importância dada a religião.
\end{enumerate}

\begin{Shaded}
\begin{Highlighting}[]
\FunctionTok{library}\NormalTok{(tidyverse)}

\CommentTok{\# Carregando e ajustando os dados}
\NormalTok{cces }\OtherTok{=} \FunctionTok{read\_csv}\NormalTok{(}\StringTok{"cces\_sample.csv"}\NormalTok{)}
\end{Highlighting}
\end{Shaded}

\begin{verbatim}
## Rows: 1000 Columns: 25
## -- Column specification --------------------------------------------------------
## Delimiter: ","
## dbl (25): caseid, region, gender, educ, edloan, race, hispanic, employ, mars...
## 
## i Use `spec()` to retrieve the full column specification for this data.
## i Specify the column types or set `show_col_types = FALSE` to quiet this message.
\end{verbatim}

\begin{Shaded}
\begin{Highlighting}[]
\NormalTok{cces}\SpecialCharTok{$}\NormalTok{edloan[}\FunctionTok{is.na}\NormalTok{(cces}\SpecialCharTok{$}\NormalTok{edloan)] }\OtherTok{=} \DecValTok{2}
\NormalTok{cces1 }\OtherTok{=} \FunctionTok{drop\_na}\NormalTok{(cces)}
\NormalTok{ex3 }\OtherTok{=}\NormalTok{ cces1 }\SpecialCharTok{\%\textgreater{}\%}
  \FunctionTok{filter}\NormalTok{(marstat }\SpecialCharTok{==} \DecValTok{1} \SpecialCharTok{\&}\NormalTok{ employ }\SpecialCharTok{==} \DecValTok{1} \SpecialCharTok{|}\NormalTok{ marstat }\SpecialCharTok{==} \DecValTok{6} \SpecialCharTok{\&}\NormalTok{ employ }\SpecialCharTok{==} \DecValTok{1}\NormalTok{) }

\NormalTok{tabela }\OtherTok{=} \FunctionTok{summarise}\NormalTok{(ex3, }\StringTok{\textasciigrave{}}\AttributeTok{média}\StringTok{\textasciigrave{}} \OtherTok{=} \FunctionTok{mean}\NormalTok{(pew\_religimp),}
                        \StringTok{\textasciigrave{}}\AttributeTok{mediana}\StringTok{\textasciigrave{}} \OtherTok{=} \FunctionTok{median}\NormalTok{(pew\_religimp))}

\FunctionTok{options}\NormalTok{(}\AttributeTok{digits =} \DecValTok{3}\NormalTok{)}
\FunctionTok{kable}\NormalTok{(tabela)}
\end{Highlighting}
\end{Shaded}

\begin{longtable}[]{@{}rr@{}}
\toprule\noalign{}
média & mediana \\
\midrule\noalign{}
\endhead
\bottomrule\noalign{}
\endlastfoot
2.22 & 2 \\
\end{longtable}

\begin{verbatim}

Esse pacote vem instalado junto com o tidyverse e é útil para a construção de gráficos na linguagem R.


``` r
# Leitura da base
dados <- read_delim( "https://www.stat.ubc.ca/~jenny/notOcto/STAT545A/examples/gapminder/data/gapminderDataFiveYear.txt", delim = "\t")
\end{verbatim}

\begin{verbatim}
## Rows: 1704 Columns: 6
## -- Column specification --------------------------------------------------------
## Delimiter: "\t"
## chr (2): country, continent
## dbl (4): year, pop, lifeExp, gdpPercap
## 
## i Use `spec()` to retrieve the full column specification for this data.
## i Specify the column types or set `show_col_types = FALSE` to quiet this message.
\end{verbatim}

\begin{Shaded}
\begin{Highlighting}[]
\NormalTok{dados1 }\OtherTok{=} \FunctionTok{filter}\NormalTok{(dados, year }\SpecialCharTok{==} \DecValTok{2007}\NormalTok{)}
\end{Highlighting}
\end{Shaded}

\section{Gráfico de barra}\label{gruxe1fico-de-barra}

\begin{Shaded}
\begin{Highlighting}[]
\FunctionTok{library}\NormalTok{(ggplot2)}
\FunctionTok{library}\NormalTok{(ggthemes)}


\FunctionTok{ggplot}\NormalTok{(dados1, }\FunctionTok{aes}\NormalTok{(}\AttributeTok{x =}\NormalTok{ continent, }\AttributeTok{fill =}\NormalTok{ continent)) }\SpecialCharTok{+} 
  \FunctionTok{geom\_bar}\NormalTok{() }\SpecialCharTok{+} 
  \FunctionTok{scale\_fill\_brewer}\NormalTok{(}\AttributeTok{palette =} \StringTok{"Blues"}\NormalTok{, }\AttributeTok{direction =} \SpecialCharTok{{-}}\DecValTok{1}\NormalTok{) }\SpecialCharTok{+}
  \FunctionTok{labs}\NormalTok{(}\AttributeTok{title =} \StringTok{"Quantidade de paises por continentes"}\NormalTok{,}
       \AttributeTok{subtitle =} \StringTok{"Dados 2007"}\NormalTok{,}
       \AttributeTok{x =} \StringTok{""}\NormalTok{,}
       \AttributeTok{y =} \StringTok{"Quantidade"}\NormalTok{,}
       \AttributeTok{fill =} \StringTok{"Continente"}\NormalTok{) }\SpecialCharTok{+}
  \FunctionTok{guides}\NormalTok{(}\AttributeTok{fill =} \StringTok{"none"}\NormalTok{) }
\end{Highlighting}
\end{Shaded}

\includegraphics{Aula04_files/figure-latex/unnamed-chunk-5-1.pdf}

\section{Bloxplot}\label{bloxplot}

É utilizado no cruzamento de uma variável qualitativa (Continente)
versus uma quantitativa (Expectativa de vida)

\includegraphics{Aula04_files/figure-latex/unnamed-chunk-6-1.pdf}

\section{O diagrama de dispersão}\label{o-diagrama-de-dispersuxe3o}

É o gráfico utilizado no cruzamento de duas variáveis quantitativas.

\includegraphics{Aula04_files/figure-latex/unnamed-chunk-7-1.pdf}

\end{document}
